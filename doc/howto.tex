\documentclass[10pt,a4paper]{article}

\parindent 0cm
\parskip 1cm

\begin{document}
\centerline{\bf X-Plot version 1.3x}
\vskip 1cm
\centerline{\bf Quick Reference Card}
\vskip 1cm
\centerline{Programmed by Martin Eskildsen}
\vskip 1cm
\centerline{\it Copyright (c) 1994--1996 by}
\vskip 0.5cm
\centerline{\it Department of Mathematics, MAT,}
\centerline{\it Technical University of Denmark}

\parskip 0.5cm

{\bf Welcome} to {\it X-Plot} for Unix! This reference card will give
you a brief introduction to the operations available in the program
and how to use them. But, as always, there's no substitute for
experiments - and we hope you will find X-Plot useful in these areas!

Finally a note about this guide: Whenever you encounter a {\it
footnote}\footnote{Such as this.}, it contains commands or hints
specific to the {\it G-bar}\ at DTU.



{\bf 1. Starting.} X-Polt is started by opening the File Manager, and
clicking the icon ``{\tt xplot}''. Alternatively you can start it from
your command shell window by typing ``{\tt \~{}/xplot \&}'' ---
provided you have the proper command set up\footnote{You can copy the
start--up script to your own directory with this command: {\tt
cp~\~{}g0141/xplot/xplot~\~{}}}



{\bf 2. Stopping.} To end the program, double-click on the little box
at the upper left corner of the Root Window (the window with the
program title written at the top), or, activate the window by clicking
its title and then press the {\tt Alt}-key, hold it down, press the
{\tt F4} key and then release both keys. Or, select Quit from the File
menu.



{\bf 3. What's on the screen.} You are presented with five windows:
\begin{itemize}
\item The Root Window which carries the name of the program {\tt
(X-Plot)}
\item The Domain Window in which you draw lines, circles, rectangles,
grids, parameter plots, freehand and concentric circles.
\item The Range Window which displays the results of the calculations.
\item The Coordinates Window which displays the coordinates of the
mouse pointer when it moves inside the Domain and Range windows.
\item The Shell window which shows different messages from the
program.
\end{itemize}
All windows can be maximized or minimized as usual.



{\bf 4. How to refresh the windows.} If the display get cluttered,
which sometimes happens if you have zoomed in a lot, you need to
refresh the windows. To do this, you should select {\it Redraw} or
{\it Refresh} from the window manager menu; if your're insecure on how
to do this, please consult the system documentation.



{\bf 5. How to enter a formula.} To enter a formula, move the mouse
pointer to the Root Window and below the menu bar. Then press the
Right mouse button, and a dialog box with the current mapping function
will appear. Then you can enter the formula, which must be a function
of the complex variable $z$. When you're done, you can press:
\begin{itemize}
\item {\tt [Ok]} to use the formula for new objects only, or
\item {\tt [Recalculate]} to use the formula on all existing objects.
\end{itemize}



{\bf 6. How to create a new object.} For all types of object, except
the parameter plot, move the mouse pointer to the Domain Window, press
the left mouse button, hold it down and move the mouse. While you drag
the mouse, some information about the new object is shown in the
Coordinates Window. When you're done release the mouse button. Then
the object will be mapped to the Range Window. If the Range Window
doesn't seem to change, it's probably because the points are outside
the window. Then you could try to {\it Autoscale} (see below) to fit
the results to the Range Window.

The procedure is different for parameter plots: Here you click once in
the domain window. This makes a dialog appear, in which you enter: the
parameter function $f(t)$, the range of $t$, and the step size of
$t$. When you then press the {\tt [Ok]} button, the parameter plot is
drawn in the Domain Window, and then mapped into the Range Window.



{\bf 7. How to change the type of object.} Move to the Domain Window
and press the right mouse button. Then a dialog box will appear. To
the left in the dialog box, you can select a type of object among the
different possibilities.



{\bf 8. How to change the number of points per object.} Move to the
Domain Window and press the right mouse button. Then move to the
dialog box and click the left mouse button at the input field to the
right of the type of object and enter a new value. Then click {\tt
[Ok]}.  (This only affects new objects.) Note that you can't enter any
values for the types Parameter Plot and Freehand.



{\bf 9. How to change the grid distances.} Move to the Domain Window
and press the right mouse button. In the dialog box, move to the Delta
area and click on the {\it Grid X} or {\it Grid Y} input field. After
editing, click {\tt [Ok]}. Note that this only affects new objects.



{\bf 10. How to change the distance between concentric circles.} Move
to the Domain Window and press the right mouse button. In the dialog
box, move to the Delta area and click on the {\it Radius} input field.
After editing, click {\tt [Ok]}. This only affects new objects.



{\bf 11. How to turn axis on or off.} Move the mouse pointer to the
window you want to change the axis setting of.  Then click the right
mouse button. When the dialog box appears, click the {\tt [Axis]}
button in the Options area.



{\bf 12. How to turn coordinates on or off.} Move the mouse pointer to
the window you want to change the coordinates setting of. Then click
the right mouse button. When the dialog box appears, click the {\tt
[Coordinates]} button in the Options area.



{\bf 13. How to zoom in or out.} This can be done in two ways: With
the dialog box or with the mouse.
\begin{itemize}
\item Move the mouse pointer to the window you want to zoom in or out
on.  Then click the right mouse button. When the dialog box appears,
click the {\tt [Zoom In]} or {\tt [Zoom Out]} button in the Options
area.
\item With the mouse: Move to the window you want to zoom on. Then
press the {\it Control} key on the keyboard and hold it down. Then
press the left or right mouse button to zoom in or out. Finally
release the {\it Control} key.
\end{itemize}



{\bf 14. How to zoom in on an area.} Move the mouse pointer to the
window you want to zoom in at. Then press the right mouse button. When
the dialog box appears, click {\tt [Zoom Area]}. Then use the mouse as
if you were creating a new object.



{\bf 15. How to autoscale.} Move the mouse pointer to the window you
want to zoom in at. Then press the right mouse button. When the dialog
box appears, click {\tt [Autoscale]}.



{\bf 16. How to center the display.} Move the mouse pointer to the
window you want to center at. Then press the {\it Shift} key on the
keyboard and hold it down. Then press the left mouse button at the
point you want centered. Finally release the {\it Shift} key.



{\bf 17. How to draw Points in the Range Window.} Move the mouse
pointer to the Range Window and press the right mouse button.  Then
click {\tt [Points]} in the drawing mode area.



{\bf 18. How to draw Lines in the Range Window.} Move the mouse
pointer to the Range Window and press the right mouse button.  Then
click {\tt [Lines]} in the drawing mode area.



{\bf 19. How to draw Directed Objects in the Domain Window.} Move the
mouse pointer to the Domain Window and press the right mouse
button. Then click {\tt [Directed Lines]} in the options area and then
click [Close]. Note that the {\tt [Param.\ as Points]} option ``wins''
over {\tt [Directed Lines]} in case of the types Iteration, Freehand
and Parameter Plot.



{\bf 20. How to draw Directed Lines in the Range Window.} Move the
mouse pointer to the Range Window and press the right mouse button.
Then click {\tt [Directed Objects]} in the drawing mode area.



{\bf 21. How to draw as points in the Domain Window.} This option is
only of interest in case of the types Iteration, Freehand and
Parameter Plot. Move the mouse pointer to the Domain Window and press
the right mouse button. Then click {\tt [Param.\ as Points]} in the
Options area.



{\bf 22. How to clear the windows.} Select {\tt New} from the {\tt
File} menu in the Root Window.



{\bf 23. How to load a library.} Select {\tt Load...} from the {\tt
File} menu in the Root Window. Note that in the current version, this
operation isn't perfectly stable; you shouldn't load the same library
over and over, as this might confuse the program.



{\bf 24. How to print.}\footnote{The printout is found in 302/43.}
Select {\tt Print} from the {\tt File} menu. This pops up a window
where the formula is written. Here you can add your own text, both
before and after the function $f(z)=\dots$\ A newline is produced by
the symbol ``{\tt \#}''; an empty line is produced by the symbol
``{\tt \#\#}''.



{\bf 25. How to change the print command.} Select {\tt Change
Printer...} from the {\tt File} menu. A window opens, where you can
write your Unix print command prefix\footnote{Such as {\tt lp
-dgps1}}.



\end{document}

